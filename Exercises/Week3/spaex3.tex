\documentclass[10pt]{article}

\usepackage[utf8x]{inputenc}
\usepackage[T1]{fontenc}
\usepackage{amsfonts,amsmath,amssymb,amsthm,booktabs,color,graphicx}
\usepackage[ruled,vlined,linesnumbered]{algorithm2e}
\usepackage{enumitem}

\title{String Processing Algorithms 2015 - Week 3 Exercises}
\author{Rodion Efremov}
\date{\today}

\begin{document}
\maketitle

\section*{Exercise 1}
\color{blue}
Describe how to modify the LSD radix sort algorithm to handle strings of varying length. The time complexity should be the one given in Theorem 1.27.
\color{black}

\subsection*{Solution}
The time complexity mentioned in the Theorem 1.27 is $\mathcal{O}(||\mathcal{R}|| + m\sigma)$. All we need to do is to modify the \textsc{Counting-Sort} procedure:

\begin{algorithm}
\SetKw{KwLet}{let}
\SetKw{KwEmptyMap}{be an empty map}
\SetKw{KwNil}{nil}
\SetKw{KwNotMapped}{is not mapped in}

\For{$i = 0$ \KwTo $\sigma - 1$}{
  $C[i] = 0$ \\
}
$s = 0$ \\
\For{$i = 1$ \KwTo $n$}{
  \If{$|S_i| < \ell$}{
    $s = s + 1$ \\  
  }
  \Else{
    $C[S_i[\ell]] = C[S_i[\ell]] + 1$ \\
  }
}
$sum = s$ \\
\For{$i = 0$ \KwTo $\sigma - 1$}{
  $tmp = C[i]$ \\
  $C[i] = sum$ \\
  $sum = sum + tmp$ \\
}
$p = 0$ \\
\For{$i = 1$ \KwTo $n$}{
  \If{$|S_i| < \ell$}{
     $J[p] = S_i$ \\
     $p = p + 1$ \\
  } 
  \Else{
    $J[C[S_i[\ell]]] = S_i$ \\
    $C[S_i[\ell]] = C[S_i[\ell]] + 1$ \\
  }
}
$\mathcal{R} = J$ \\
\caption{\textsc{Counting-Sort}$(\mathcal{R} = \{ S_1, S_2, \dots, S_n \}, \ell)$}
\end{algorithm}
Now, the desired LSD radix sort for variable-length strings is
\begin{algorithm}
$m = \max_{i = 1, 2, \dots, n} \{ |S_i| \}$ \\
\For{$\ell = m$ \KwTo $1$}{
  \textsc{Counting-Sort}$(\mathcal{R}, \ell)$ \\
}
\caption{LSDRadixSort($\mathcal{R} = \{ S_1, S_2, \dots, S_n \}$)}
\end{algorithm}

\section*{Exercise 2}
\color{blue}
Use the lcp comparison technique to modify the standard insertion sort algorithm so that it sorts strings in $\mathcal{O}(\Sigma LCP(\mathcal{R}) + n^2)$ time.
\color{black}

\subsection*{Solution}
\begin{algorithm}
\SetKw{KwAnd}{and}
\For{$i = 2$ \KwTo $n$}{
	$s = S_i$ \\
	$j = i - 1$ \\
	\If{$\text{LCP}_{\mathcal{R}}[i - 1] = 0$}{
	   $\text{LCP}_{\mathcal{R}}[i] = 0$ \\
	}
	\While{$j > 0$ \KwAnd \textsc{lcpCompare}$(S_j, s, \text{LCP}_{\mathcal{R}}[i]) > 0$}{
	  $S_{j + 1} = S_j$ \\
	  $j = j - 1$ \\
	}
	$S_{j + 1} = s$ \\
}
\caption{\textsc{Insertionsort}$(\mathcal{R}, \text{LCP}_{\mathcal{R}})$}
\end{algorithm}
\subsubsection*{Why do we set LCP to zero?}
Consider the following:
\begin{center}
\begin{tabular}{|l|l|l|}
\hline
$i$ & $S_i$ & $\text{LCP}_{\mathcal{R}}[i]$ \\
\hline
1 & $aaaba$ & 0 \\
2 & $abaaa$ & 1 \\
3 & $aaa$     & 1 \\
4 & $baaa$   & 0 \\
5 & $aabba$ & 0 \\
6 & $abaaa$ & 1 \\
\hline
\end{tabular}
\end{center}

\section*{Exercise 3}
\color{blue}
Give an example showing that the worst case time complexity of string binary search without precomputed lcp information is $\Omega(m \log n)$.
\color{black}

\subsection*{Solution}
Suppose the sorted list of strings is 
\begin{align*}
\langle \overbrace{aaa \dots aaa}^{m}a_1, \\
\overbrace{aaa \dots aaa}^{m}a_2, \\
\dots \\
\overbrace{aaa \dots aaa}^{m}a_n \rangle
\end{align*}
and the string to search for is $\overbrace{aaa \dots aaa}^{m}a'$, where $a' \neq a_i$ for any $i \in \{ 1, 2, \dots, m \}$ and $a_1 \leq  a_2 \leq \dots \leq a_n$. Now as there is no match, the binary search will do $\Omega(\log n)$ string comparisons, and as we assume the naïve implementation of the string comparison, each comparison will have to iterate through $m$ first characters $a$ before getting to the last character that fails the search, which leads to the worst case time complexity of $\Omega(m \log n)$.

\section*{Exercise 4}
\color{blue}
Let $S[0..n)$ be a string over an integer alphabet. Show how to build a data structure in $\mathcal{O}(n)$ time and space so that afterwards the Karp-Rabin hash function $H(S[i..j))$ for the factor $S[i..j)$ can be computed in constant time for any $0 \leq i \leq j \leq n$.
\color{black}

\subsection*{Solution}
The actual query routine follows
\begin{algorithm}
$l = j - i$ \\
$f = F[l]$ \\
$a = E[j]$ \\
$b = E[i]$ \\
$r = (a - bf) \mod q$ \\
\If{$r < 0$}{
  \KwRet $r + q$ \\
}
\KwRet $r$ \\
\caption{\textsc{Query}$(E[0..n], F[0..n], i, j, q)$}
\end{algorithm}
The preprocessing step is
\begin{algorithm}
$E[0..n] = \overbrace{[0, 0, \dots, 0]}^{n + 1}$ \\
$F[0..n] = \overbrace{[0, 0, \dots, 0]}^{n + 1}$ \\
$m = 1$ \\
\For{$i = 1$ \KwTo $n$}{
  $E[i] =$ \textsc{Compute-Entry}$(S, i, m)$ \\
  $m = m * r$ \\
}
$f = 1$ \\
\For{$i = 0$ \KwTo $n$}{
  $F[i] = f$ \\
  $f = f * r$ \\
}
\KwRet $(E, F)$ \\
\caption{\textsc{Preprocess}$(S[0..n), r, q)$}
\end{algorithm}
\end{document}